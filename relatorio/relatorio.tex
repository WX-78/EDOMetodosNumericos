\documentclass[
		12pt,            %tamanho padrão da fonte
		a4paper
	      ]  
	      {article}          %tipo de documento
	      
%----------pacotes----------
\usepackage[brazilian=nohyphenation]{hyphsubst} % elimina a criação de hifens entre linhas
\usepackage{amsmath}			% equações
\usepackage{graphicx}			% imagens
\usepackage[brazilian]{babel}		% tradução para PT BR
\usepackage[utf8]{inputenc}		% acentos
\usepackage{indentfirst}		% insere o primeiro parágrafo
\usepackage{setspace}			% permite escolher espaço entre linhas	

%----conf. da aparencia-----
	\frenchspacing						% better looking spacing
	%\renewcommand{\rmdefault}{phv} % Arial
	%\renewcommand{\sfdefault}{phv} 
	\renewcommand{\baselinestretch}{1.5} % espaço entre linhas
	\usepackage{titlesec}
		\titleformat*{\section}{\large\bfseries}

\usepackage{microtype}  		% para melhorias de justificação
\usepackage{setspace}			% permite escolher espaço entre linhas	
	% O tamanho do parágrafo é dado por:
	\setlength{\parindent}{13mm}

	% Controle do espaçamento entre um parágrafo e outro:
	\setlength{\parskip}{5mm}  % tente também \onelineskip

\usepackage[a4paper, left=30mm,right=20mm,top=30mm,bottom=20mm,%
            footskip=.30mm]{geometry}

\usepackage{caption}
\captionsetup[figure]{labelformat=simple}%
\renewcommand{\thefigure}{\arabic{figure}}
\usepackage[font=small,labelfont=bf]{caption}


\usepackage{graphicx}    %graficos

\usepackage{xcolor} % for setting colors
\usepackage{listings} % codigo fonte
\usepackage{authblk}

\definecolor{codegreen}{rgb}{0,0.6,0}
\definecolor{codegray}{rgb}{0.5,0.5,0.5}
\definecolor{codepurple}{rgb}{0.58,0,0.82}
\definecolor{codeblue}{rgb}{0,0,139}
\definecolor{backcolour}{rgb}{0.95,0.95,0.92}
 
\lstdefinestyle{mystyle}{
	backgroundcolor=\color{backcolour},   
        commentstyle=\color{codegreen},
	keywordstyle=\color{codeblue},
	numberstyle=\tiny\color{codegray},
	stringstyle=\color{codepurple},
	basicstyle=\footnotesize,
	breakatwhitespace=false,         
	breaklines=true,                 
	captionpos=b,                    
	keepspaces=true,                 
	numbers=left,                    
	numbersep=5pt,                  
	showspaces=false,                
	showstringspaces=false,
	showtabs=false,                  
	tabsize=2
}

\lstset{style=mystyle}

%----cabeçalho e rodapé-----
\usepackage{fancyhdr}
               
	\pagestyle{fancy}
	\rhead{}
	\chead{}
	\lhead{\small Universidade Federal do Rio Grande \hfill \thepage}
	\lfoot{}
	\cfoot{}
	\rfoot{}
	\renewcommand{\headrulewidth}{0.05pt}
	\renewcommand{\footrulewidth}{0.0pt}
	\thispagestyle{empty}

	\title{Métodos numéricos aplicados a equações diferenciais ordinárias}
	\newcommand{\authorA}{André Lopes Brum}
	\newcommand{\authorB}{Pedro Henrique Fernandes Lobo}
	\newcommand{\authorC}{Vinícius Heidtmann Avila}

	\usepackage[hang, perpage]{footmisc}
	\usepackage{lipsum}

	\addtolength{\footskip}{0.5cm}
	\setlength{\footnotemargin}{0.3cm}
	\setlength{\footnotesep}{0.4cm} 
	\setlength{\skip\footins}{3mm}


\begin{document}

	%-----------capa------------
	\begin{titlepage}
	\makeatletter

	\noindent% just to prevent indentation narrowing the line width for this line
		\includegraphics[height=25.0mm]{logoFurg.png}
		\begin{minipage}[b]{0.7\textwidth}
			\centering
			{\Large Universidade Federal do Rio Grande} \\
			\vspace{1.5mm}
			{Instituto de Matemática, Estatística e Física} \\
			\vspace{0.5mm}
			{\sc Pós-Graduação em Modelagem Computacional} \\
			\vspace{0.2mm}
			\centering\rule{0.95\textwidth}{1pt}
		\end{minipage}
		\includegraphics[height=30.0mm, clip, trim = 10mm  0 0 0]{logoIMEF.jpg}
		\begin{center}	
			{\bf\small \MakeUppercase{\authorA}}\\		 
			{\bf\small \MakeUppercase{\authorB}}\\		 
			{\bf\small \MakeUppercase{\authorC}}\\
		\end{center}
			\vspace{5cm}
		\begin{center}
			{\bf\large \MakeUppercase{\@title}} \\
		\end{center}
		\vfill		
		\begin{center}
			\vspace{15mm}
			Rio Grande-RS, \the\year 
		\end{center}

	\makeatother
	\end{titlepage}


	%------folha de rosto-------	

		\begin{titlepage}	
	
	\makeatletter

	\noindent% just to prevent indentation narrowing the line width for this line
		\includegraphics[height=25.0mm]{logoFurg.png}
		\begin{minipage}[b]{0.7\textwidth}
			\centering
			{\Large Universidade Federal do Rio Grande} \\
			\vspace{1.5mm}
			{Instituto de Matemática, Estatística e Física} \\
			\vspace{0.5mm}
			{\sc Pós-Graduação em Modelagem Computacional} \\
			\vspace{0.2mm}
			\centering\rule{0.95\textwidth}{1pt}
		\end{minipage}
		\includegraphics[height=30.0mm, clip, trim = 10mm  0 0 0]{logoIMEF.jpg}
		\begin{center}	
			{\bf\small \MakeUppercase{\authorA}}\\		 
			{\bf\small \MakeUppercase{\authorB}}\\		 
			{\bf\small \MakeUppercase{\authorC}}\\
		\end{center}
			\vspace{5cm}
		\begin{center}
			{\bf\large \MakeUppercase{\@title}} \\
		\end{center}
	
			\vspace{10mm}
			
			\begin{flushright}
			\begin{minipage}[t]{0.5\textwidth}
				\setlength{\parindent}{0mm}
				Trabalho apresentado ao programa de Pós-Graduação em Modelagem 
				Computacional, pertencente a Universidade Federal do Rio Grande, 
				como requisito de avaliação parcial para conclusão do curso de 
				Métodos Numéricos Aplicados.
			\end{minipage}	
			\end{flushright}

			\vspace{15mm}
			\vfill	
			\begin{center}
				Rio Grande-RS, \the\year 
			\end{center}
			\makeatother
	
		\end{titlepage}
		\cleardoublepage

	
		\section{Objetivo}
	
	Comparar os métodos de Euler e Runge-Kutta de dois estágios
	ao resolver uma equação diferencial ordinária de segunda ordem.




	\section{Introdução}

	Muitos problemas em ciências e engenharias são modelados através de equações 
	diferenciais. Porém, nem sempre elas apresentam soluções analíticas. Por isto, 
	muitas vezes é necessário recorrer ao computador para que encontrar as soluções 
	numéricas.\par

	Desta maneira, para compreender a lógica por trás dos métodos Runge Kutta e de Euler, 
	que são bastante conhecidos, iremos aplicá-los em um problema simples e recorrente 
	na física, fazendo uma breve discusão sobre eficácia a deles.

	\section{Fundamentação teórica}

	Seja uma equação diferencial da forma:
	
	\vspace{-5mm}
	\begin{equation}\label{eq:diff}
		\begin{aligned}
			\frac{d}{dt} y(t)&=f\left ( t,y(t) \right ) \\
			y(0) &= y_0
		\end{aligned}
	\end{equation}
	\vspace{-5mm}
	
	Embora esta equação possa ter uma solução analítica, ou seja, capaz de estabelecer 
	diferenciais contínuas em um intervalo, o computador pode imitar o processo de 
	diferenciação se o espaço for discretizado. Para isto, vamos procurar uma solução através 
	de uma expansão em série de Taylor:
	
	\vspace{-13mm}
	\begin{equation*}
		y(t+\Delta t) = y(t) + \Delta t \dfrac{d}{dt}y(t)
		+  \frac{\Delta t }{2!}\dfrac{d^2}{dt^2}y(t) + 
		\frac{ \Delta t }{3!}  \dfrac{d^3}{dt^3}y(t) +...+
		\frac{ \Delta t }{n!}  \dfrac{d^n}{dt^n}y(t) 
	\end{equation*}
	\vspace{-13mm}

	Se truncarmos a série no primeiro termo e isolarmos a diferencial obtemos:
	
	\vspace{-5mm}
	\begin{equation}\label{eq:diffA}
		\frac{d}{dt}y(t) = \frac{y(t + \Delta t)- y(t)}{\Delta t} + \mathcal{O}(\Delta t)
	\end{equation}
	\vspace{-7mm}
	
	Com isso, comparando as equações \ref{eq:diff} e \ref{eq:diffA} e negligenciando 
	os demais termos a partir do segundo, podemos determinar de uma forma iterativa, qual o valor da função 
	$f\left ( t,y(t) \right )$ no ponto seguinte distante de $\Delta t$. Desta forma, a expressão 
	seguinte é conhecida como método de Euler e o cálculo da função avaliada na posição subsequente 
	é dada por:
	
	\vspace{-5mm}
	\begin{equation}
		y(t + \Delta t) \approx y(t) + \Delta t f \left ( t, y \left ( t \right ) \right )
	\end{equation}
	\vspace{-7mm}
	
	Agora iremos reescrever a equação anterior, substituindo $y_n$ por $y(t)$; $t$ por $t_n$ e 
	$\Delta t$ por $h$. Logo, o método de Euler é expresso na forma a seguir para obtermos uma 
	aproximação discretizada da solução $y_{n+1}$ avaliada em cada ponto $n + 1$.

	\vspace{-5mm}
	\begin{equation}\label{eq:euler}
		y_{n+1} = y_n + h f \left ( t_n, y_n \right ) 
	\end{equation}
	\vspace{-7mm}

	Obviamente para utilizarmos o método é necessário fornecer uma condição inicial para o cálculo 
	do primeiro ponto. Além disso, há um erro associado ao truncamento realizado que vai aumentando
	a cada iteração. Portanto, para melhorar a aproximação, devemos adotar outro procedimento. Isto 
	pode ser realizado através de uma quadratura numérica (ver \cite{NASA1967, Chapman2010C6}) que 
	resulta na família de métodos Runge Kutta.\par
	
	Para isto, há varias maneiras de estabelecer a função que resolve numericamente a EDO e que podem 
	ser verificados em \cite{NASA1967,Chapman2010C7, Hairer1989}. Mas este assunto vai muito além da 
	escopo deste trabalho e por isso vamos simplesmente apresentar a equação com parâmetros arbitrados 
	que o faz ser conhecido como método {\it midpoint} Runge-Kutta de dois estágios.


	\vspace{-5mm}
	\begin{equation}\label{eq:rk2}
		\begin{aligned}
			k_1 &= hf(t_i,y_i)\\
			k_2 &= hf\left (t_i+\frac{h}{2}, y_i + \frac{h}{2} k_1 \right )\\
			y_{i+1} &= y_i + hk_2
		\end{aligned}
	\end{equation}
	\vspace{-5mm}

	\section{Aplicação}
	
	Muitos problemas físicos apresentam comportamento oscilatório como sistemas massa-mola, pêndulo simples,
	pêndulo de torção, circuito LC, oscilador harmônico quântico, etc. Em todos estes casos, quando não há  
	termos responsáveis pela restauração e dissipação de energia, o sistema entra no regime de movimento 
	harmônico simples\footnote{Em alguns casos, como o do pêndulo simples, a equação diferencial costuma passar
	por um processo de linearização para que possa ser considerado movimento harmônico simples.}, ou seja, 
	passa a oscilar eternamente. Matematicamente as equações que governam este tipo de fenômeno tem a forma 
	apresentada a seguir:

	\vspace{-5mm}
	\begin{equation}\label{eq:geral}
		\frac{d^2 }{dt^2} \psi(t) = - \omega^2 \psi(t)
	\end{equation}

	Onde o termo $\omega$ é definido como a frequência natural do sistema, $t$ é variável independente
	e $\psi(t)$ é o termo dependente. Note também que é conveniente aplicar uma redução de ordem sobre a 
	diferencial para utilização dos métodos numéricos. Isto porque eles foram elaborados para serem utilizados 
	em sistemas de equações diferenciais de primeira ordem. Sendo assim, reescrevendo a expressão anterior, 
	obtemos:

	\vspace{-5mm}
	\begin{equation}\label{eq:reduzida}
		\begin{aligned}			
			\frac{dz}{dt} &= - \omega^2 \psi \\
			\frac{d\psi}{dt} &= z
		\end{aligned}
	\end{equation}
	
	As equações \ref{eq:geral} e \ref{eq:reduzida} apresentam a solução dada por \ref{eq:solucao}, onde 
	$A$ é a amplitude de oscilação e $\phi$ é o ângulo de fase. Ambos os termos são arbitrados como
	condições iniciais do problema.

	\vspace{-10mm}
	\begin{equation}\label{eq:solucao}
		\psi(t)=A\cos(\omega t + \phi)
	\end{equation}
	\vspace{-15mm}

	\section{Simulações}

	Em geral, métodos numéricos são normalmente utilizados quando é dificil encontrar soluções analíticas 
	de EDOs. Entretanto, no nosso caso esta é conhecida, então é  conveniente fazer uma comparação entre ela 
	e os métodos Runge-Kutta e de Euler. Para isto, vamos estabelecer as condições de contorno do problema. 
	Desta maneira, derivando a equação \ref{eq:solucao} em relação a $t$, obtemos $z(t)$ apresentada a seguir:
	
	\vspace{-5mm}
	\begin{equation}\label{eq:derivada}
		\frac{d}{dt}\psi(t)= z(t) = -A \omega \sin(\omega t + \phi)
	\end{equation}

	Como $A$ e $\omega$ são positivos, ao inspecionar as equações \ref{eq:solucao} e \ref{eq:derivada} 
	e arbitrar $\psi(0) = 1$ e $z(0) = 0$, implica em $\phi = 0$ e $A = 1$. Além disso, adotamos a constante 
	$\omega = 1.4$. E com isso, passamos a determinar todos os termos presentes na solução dada pela equação 
	\ref{eq:solucao}.\par
	
	Para resolver numericamente a EDO, construímos dois programas em {\it Fortran90}, capazes de resolvê-la 
	pelo método de Euler e Runge-Kutta de dois estágios ao aplicar as relações \ref{eq:euler} e \ref{eq:rk2} 
	respectivamente. Cada programa utiliza o primeiro ponto obtido pelas condições de contorno e a partir dele, 
	a cada iteração, estabelece o valor do ponto seguinte. Obviamente foi necessário reduzir a ordem da derivada 
	(ver a relação \ref{eq:reduzida}) e executar o algorítimo simultaneamente nas duas diferenciais. O leitor 
	pode ver mais detalhes através códigos fontes apresentados em anexo ou no repositório disponível em \cite{projeto}.
	
	\section{Resultados e discussões}
	
	Ao compilar\footnote{O compilador utilizado é o {\it gfortran} 5.4.0} e executar os programas, o 
	terminal\footnote{{\it Linux Lite} 3.6} mostra três colunas que representam $t$, $\psi(t)$ e $z(t)$ 
	respectivamente. Ao plotarmos um gráfico de $\psi(t)$ em função de $t$ contendo a curva teórica dada 
	pela equação \ref{eq:solucao} e os dois métodos numéricos, obtemos as figuras \ref{fig:A} e \ref{fig:B} que 
	diferem apenas pelo tamanho do passo $h$.

	Na figuras \ref{fig:A} e \ref{fig:B}, ao compararmos as soluções numéricas com a curva teórica, verificamos 
	que parecem haver defasagens e um aumento progressivo nas amplitudes. Mas, na verdade, isto ocorre porque a 
	cada iteração os métodos vão acumulando erros que aumentam com o tamanho do passo $h$. Ao diminuirmos este,
	o resultado numérico passa a se aproximar mais da solução analítica. 

	\begin{figure}[h!]
		\centering
		\captionsetup{width=0.75\textwidth}
		\resizebox{0.65\textwidth}{!}{\input{graficoA}}
		\caption{Solução numérica através dos métodos de Euler e Runge-Kutta de dois estágios seguido da
			 analítica dada pela equação \ref{eq:solucao}. Onde $A = 1$, $\phi = 0$ e $\omega = 1.4$.
			 O tamanho do passo arbitrado é $h = 0.12$.}
		\label{fig:A}
	\end{figure}

	Fazendo uma comparação entre os métodos numéricos, verificamos que o de Euler se afasta da solução analítica 
	com muito mais rapidez quando comparado com o Runge-Kutta. Isto ocorre porque a forma com que estes 
	algorítimos são construídos levam em conta uma aproximação por quadratura numérica, fazendo com que o método 
	de Euler se torne um caso particular (e de menor acurácia) da familia de métodos Runge-Kutta de qualquer 
	estágio superior (ver \cite{NASA1967, Chapman2010C6, Chapman2010C7}).
	
	\begin{figure}[h!]\label{fig:hPequeno}
		\centering
		\captionsetup{width=0.75\textwidth}
		\resizebox{0.65\textwidth}{!}{\input{graficoB}}
		\caption{Solução numérica através dos métodos de Euler e Runge-Kutta de dois estágios seguido da
			 analítica dada pela equação \ref{eq:solucao}. Onde $A = 1$, $\phi = 0$ e $\omega = 1.4$.
			 O tamanho do passo arbitrado é $h = 0.05$.}
		\label{fig:B}
	\end{figure}
	Diante do exposto, ao realizarmos as comparações, consideramos que o método Runge-Kutta de dois estágios é 
	mais eficaz, pois apresenta uma melhor aproximação da solução analítica. Entretanto, este possui uma equação 
	adicional para ser resolvida a cada iteração e certamente ele exige um maior esforço computacional.
	
	\cleardoublepage
	\renewcommand{\baselinestretch}{1.5} % espaço entre linhas
	\nocite{*}
	\bibliographystyle{abntex2-num}
	\bibliography{referencia.bib} 

	\cleardoublepage
	\section*{Anexo A}
	\renewcommand{\lstlistingname}{Listagem}

	\renewcommand{\baselinestretch}{1.0} % espaço entre linhas
	\lstinputlisting[language=Fortran, caption={Código fonte em {\it Fortran90} para o método Runge-Kutta 
			de dois estágios.}]{../programas/rk2.f90}

	\cleardoublepage
	\section*{Anexo B}
	\lstinputlisting[language=Fortran, caption={Código fonte {\it Fortran90} para o método de Euler.}]
		        {../programas/euler.f90}
	
	\cleardoublepage
	\end{document}
